\documentclass[
	twocolumn,
	12pt,				% tamanho da fonte
	twoside,			% para impressão em frente e verso. Oposto a oneside ---- Este comando ajusta automaticamente as margens
	a4paper,			% tamanho do papel. 
	english,			% idioma adicional para hifenização
	brazil				% o último idioma é o principal do documento
	]{article}

% Compilador de arquivos .bib
\usepackage[
	backend=biber,
	style=numeric-comp,
	sorting=none,
	]{biblatex}

% Insira arquivo .bib da bibliografia aqui:
\addbibresource{bibliografia.bib}

\usepackage[portuguese]{babel}
\usepackage[utf8]{inputenc}	
\usepackage{hyphsubst}
\usepackage{lipsum} 
\usepackage{fancyhdr}
\usepackage{multicol}
\usepackage{indentfirst}
\usepackage{graphicx}
\usepackage{float}
\usepackage[skip=2pt,font=scriptsize]{caption}
\usepackage{listings}
\usepackage{algorithm, algpseudocode}

% ----
% Ajusta as margens
% ----

\usepackage{geometry}
\geometry{
	a4paper,
	left=3cm,
	top=3cm,
	bottom=2cm,
	right=2cm,
}

% ----
% Ajusta o cabeçalho e o rodapé
% ----

\pagestyle{fancy}
\fancyhf{}
\rhead{Samuel Brum Martins}
\lhead{TP 1 - Matemática Discreta}
\rfoot{Página \thepage}

\setlength\columnsep{30pt}

\title{Passeio do Cavalo - Heurística de Warnsdoff}
\author{Samuel Brum Martins}
\date{}

% ----
% Início do documento
% ----

\begin{document}

% Seleciona o idioma do documento 
% (conforme pacotes do babel)
% ---------------------------------
%\selectlanguage{english}
\selectlanguage{brazil}
% ---------------------------------

% Retira espaço extra obsoleto entre as frases.
\frenchspacing 

% Primeira página inclui capa e texto
{\let\newpage\relax\maketitle}
\thispagestyle{fancy}

% Escreva aqui \section ou \chapter
\section*{Introdução}
O problema do passeio do cavalo é um quebra cabeças milenar,
útil como exemplo e simplificação de problemas em teoria de grafos.

O problema original consiste em percorrer, com o cavalo,
todo o tabuleiro de xadrez, passando apenas uma vez por cada casa.
Podem existir mais de uma solução para uma casa inicial específica e,
em alguns tabuleiros menores, podem haver casas sem uma única solução.

O algorítmo de \textit{backtracking} consiste em uma abordagem de força bruta
para solucionar o problema, essecialmente realizando "caminhadas cegas"
pelo tabuleiro e retrocedendo um passo quando não há mais movimentos.
Tal abordagem é extremamente ineficiente, já que o problema cresce
exponencialmente com o tamanho do tabuleiro, pois cada passo possui
multiplos destrichamentos, cada qual com suas multiplas n-furcações.

Uma forma de contornar tal dificuldade é através de heurísticas,
regras gerais que abrem mão de algum aspecto de uma solução completa,
como precisão, optimização, ou completeza por velocidade. 

Uma Heurística notável para o problema do passeio do cavalo é a de Warnsdoff, que consiste
em escolher o próximo movimento baseado no número de movimentos possíveis
da casa.

\section*{Implementação}
A específicação do projeto requeria que uma biblioteca \verb|passeio.h| possuisse
uma função \verb|void passeio(int x, int y)| que, ao ser chamada, realizasse
o algorítmo de \textit{backtracking} para o passeio do cavalo em um 
tabuleiro 8x8, começando da casa de linha $x$ e coluna $y$.

A função passeio como foi implementada serve como casca para a chamada
da função, onde são ajustados todos os parâmetros necessários, como um
tabuleiro vazio e uma tabela de consulta para implementar a heurística
de Warnsdoff. Em seguida, é chamada a função \verb|passo|, que realiza o grosso
da implementação. Nela é realizado iterativamente o algorítmo de \textit{backtracking}
a partir da posição inicial da chamada de \verb|passeio|. Em pseudocódigo
a ideia é apresentada no Algorítmo 1.

\begin{algorithm}
	\caption{Algorítmo de backtracking para passeio do cavalo}
	\begin{algorithmic}[1]
		\Function{Backtracking}{$casa, turno$}
			\For{$movimento$ in $movimento \space validos$}
				\State $tabuleiro[casa] \gets turno$
				\If{$Backtracking(casa + movimento, turno + 1$)}
					\State \Return True
				\Else 
				\State $RetocedeCasa()$
				\EndIf	
			\EndFor
		\EndFunction
	\end{algorithmic}
\end{algorithm}

Mesmo assim, essa implementação ingênua apresenta um problema de performânce,
uma vez que a implementação possui $\mathcal{O}(8^{n^2})$, já que para cada 
casa há até 8 movimentos que o cavalo pode fazer e cada permutação de visitas
configura um passeio diferente.

Sendo assim, implementou-se também a heurística de Warnsdoff, que consiste
em ordenar os movimentos possíveis dos quadrados que possuem o menor número
de movimentos válidos até o maior número de movimentos válidos. O ordenamento 
deu-se por \verb|BubbleSort|, já que o número de elementos a serem ordenados
era sempre 8, não justificando a implementação de um sorteamento mais sofisticado. 

O conceito é intuitivo, já que as casas que possuem o menor número de movimentos válidos
também são as que tem menos pontos de acesso, e portanto são mais prováveis
de apresentarem gargalos nas tentativas e erro. Priorizando o acesso a essas
casas, reduzimos a chance de termos que retroceder um passo.

\section*{Conclusões}
Na implementação ingênua, o programa é até capaz de rodar algumas posições
iniciais em tempo hábil, com o melhor caso tendo sido em torno de 30 segundos 
e com casos nos quais o programa rodava por mais de 10 minutos sem sucesso.

Após implementação da heurística o caso médio caiu para frações de segundo, 
com os piores casos na ordem de alguns poucos segundos, claramente uma melhoria
de várias ordens de magnitude, demonstrando que, em termos práticos, uma solução
incompleta pode ser boa o suficiente, a depender da aplicação pretendida.

Os resultados com o formato de saída exigidos e o tempo de cada uma das 
soluções encontram-se no arquivo \textbf{saida.txt}, seguidas do tempo de execução de cada
uma delas, em ordem, ambos gerados pelo teste \textbf{main.c}. Ambos arquivos encontram-se na pasta "Apêndices".


\printbibliography
\end{document}